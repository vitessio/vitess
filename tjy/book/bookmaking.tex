\chapter{Making of this book}
\label{chap:bookmaking}

\section{Where to download the book source}
\label{sec:srcinstall:unixlike}

This is book version hobbit manpages.



If you are using a Unix-like system that has a sufficiently recent
version of Python (2.3~or newer) available, it is easy to install
Mercurial from source.

\begin{enumerate}
\item Download a recent source tarball from
  \url{http://www.selenic.com/mercurial/download}.
\item Unpack the tarball:
  \begin{codesample4}
    gzip -dc mercurial-\emph{version}.tar.gz | tar xf -
  \end{codesample4}
\item Go into the source directory and run the installer script.  This
  will build Mercurial and install it in your home directory.
  \begin{codesample4}
    cd mercurial-\emph{version}
    python setup.py install --force --home=\$HOME
  \end{codesample4}
\end{enumerate}
Once the install finishes, Mercurial will be in the \texttt{bin}
subdirectory of your home directory.  Don't forget to make sure that
this directory is present in your shell's search path.

You will probably need to set the \envar{PYTHONPATH} environment
variable so that the Mercurial executable can find the rest of the
Mercurial packages.  For example, on my laptop, I have set it to
\texttt{/home/bos/lib/python}.  The exact path that you will need to
use depends on how Python was built for your system, but should be
easy to figure out.  If you're uncertain, look through the output of
the installer script above, And see where the contents of the
\texttt{mercurial} directory were installed to.

\section{Book Revision History}
% http://www.andy-roberts.net/misc/latex/latextutorial4.html
%\begin{table}[!bp]
\begin{table}
\caption{Book Revision History.} \label{BookRevisionHistory}

\begin{tabular}{|l|l|l|}
\hline
\multicolumn{3}{|c|}{People contribute to this book} \\
\hline
Name  & Date  & Contribution  \\ \hline
\multirow{2}{*}{Henrik Storner} & Winter 2002  & hobbit manpages created \\
 & Winter 2008  & update hobbit manpages to become Xymon manpages \\ \hline
\multirow{2}{*}{T.J. Yang} & 12/16/2009 & convert troff file to \LaTeX{}\\
 & 12/16/2009 & Use \href{http://projects.gnome.org/dia/}{Dia} to draw Xymon Architecture diagram. \\ \hline
\end{tabular}
\end{table}

\section{Xymon Revision History}
% http://www.andy-roberts.net/misc/latex/latextutorial4.html
\begin{table}
\caption{Xymon Revision History.} \label{XymonRevisionHistory}

\begin{tabular}{|l|l|l|}
\hline
\multicolumn{3}{|c|}{People contribute to Xymon} \\
\hline
Name  & Date  & Contribution  \\ \hline
\multirow{4}{*}{Defenders} & LB & Lucus Radebe \\
 & DC & Michael Duberry \\
 & DC & Dominic Matteo \\
 & RB & Didier Domi \\ \hline
\multirow{3}{*}{Midfielders} & MC & David Batty \\
 & MC & Eirik Bakke \\
 & MC & Jody Morris \\ \hline
Forward & FW & Jamie McMaster \\ \hline
\multirow{2}{*}{Strikers} & ST & Alan Smith \\
 & ST & Mark Viduka \\
\hline
\end{tabular}
\end{table}



%%%\begin{table}[!bp]
%%%
%%%\caption{Book Revision History.} \label{BookRevisionHistory}
%%%
%%%\begin{description}
%%%\item [Winter 2002,Henrik Storner] Original author of hobbit manpages and notes,howto articles \ldots
%%%  \index{Henrik Storner}
%%%\item [12/16/2009,T.J. Yang] Editing above manpages and miscellaneous documentsinton into a book format using Latex.
%%%\item [12/16/2009,T.J. Yang] Draw architecture diagram using Dia to help illustrate the idea.
%%%  \index{T.J. Yang}
%%%\end{description}
%%%\end{table}

%%% Local Variables: 
%%% mode: latex
%%% TeX-master: "00book"
%%% End: 

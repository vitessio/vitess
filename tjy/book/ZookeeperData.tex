
\section{Zookeeper Data}\hypertarget{zookeeper-data}{}\label{zookeeper-data}

This document describes the information we keep in zookeeper, how it is generated, and how the python client uses it.

\subsection{Keyspace / Shard / Tablet Data}\hypertarget{keyspace--shard--tablet-data}{}\label{keyspace--shard--tablet-data}

\subsubsection{Keyspace}\hypertarget{keyspace}{}\label{keyspace}

Each keyspace is now a global zookeeper path, with sub-directories for its shards and action / actionlog. The Keyspace
object there contains very basic information.

```go
// see go/vt/topo/keyspace.go for latest version
type Keyspace struct \{
        // name of the column used for sharding
        // empty if the keyspace is not sharded
        ShardingColumnName string

\begin{verbatim}    // type of the column used for sharding
    // KIT_UNSET if the keyspace is not sharded
    ShardingColumnType key.KeyspaceIdType

    // ServedFrom will redirect the appropriate traffic to
    // another keyspace
    ServedFrom map[TabletType]string
\end{verbatim}
\}
```

{\tt 
\$ zk ls /zk/global/vt/keyspaces/ruser
action
actionlog
shards
}

The path and sub-paths are created by 'vtctl CreateKeyspace'.

We use the action and actionlog paths for locking only, no process is actively watching these paths.

\subsubsection{Shard}\hypertarget{shard}{}\label{shard}

A shard is a global zookeeper path, with sub-directories for its action / actionlog, and a node for some more data and replication graph.

```go
// see go/vt/topo/shard.go for latest version
// A pure data struct for information stored in topology server.  This
// node is used to present a controlled view of the shard, unaware of
// every management action. It also contains configuration data for a
// shard.
type Shard struct \{
        // There can be only at most one master, but there may be none. (0)
        MasterAlias TabletAlias

\begin{verbatim}    // This must match the shard name based on our other conventions, but
    // helpful to have it decomposed here.
    KeyRange key.KeyRange

    // ServedTypes is a list of all the tablet types this shard will
    // serve. This is usually used with overlapping shards during
    // data shuffles like shard splitting.
    ServedTypes []TabletType

    // SourceShards is the list of shards we're replicating from,
    // using filtered replication.
    SourceShards []SourceShard

    // Cells is the list of cells that have tablets for this shard.
    // It is populated at InitTablet time when a tabelt is added
    // in a cell that is not in the list yet.
    Cells []string
\end{verbatim}
\}

// SourceShard represents a data source for filtered replication
// accross shards. When this is used in a destination shard, the master
// of that shard will run filtered replication.
type SourceShard struct \{
        // Uid is the unique ID for this SourceShard object.
        // It is for instance used as a unique index in blp\_checkpoint
        // when storing the position. It should be unique whithin a
        // destination Shard, but not globally unique.
        Uid uint32

\begin{verbatim}    // the source keyspace
    Keyspace string

    // the source shard
    Shard string

    // The source shard keyrange
    // If partial, len(Tables) has to be zero
    KeyRange key.KeyRange

    // The source table list to replicate
    // If non-empty, KeyRange must not be partial (must be KeyRange{})
    Tables []string
\end{verbatim}
\}

```

{\tt 
\$ zk ls /zk/global/vt/keyspaces/ruser/shards/10-20
action
actionlog
nyc-0000200278
}

We use the action and actionlog paths for locking only, no process is actively watching these paths.

{\tt 
\$ zk cat /zk/global/vt/keyspaces/ruser/shards/10-20
\{
  "MasterAlias": \{
    "Cell": "nyc",
    "Uid": 200278
  \},
  "KeyRange": \{
    "Start": "10",
    "End": "20"
  \},
 "Cells": [
    "oe",
    "yh"
 ]
\}
}

The shard path and sub-directories are created when the first tablet in that shard is created.

The Shard object is changed when we add tablets in unknown cells, or when we change the master.

\subsubsection{Tablet}\hypertarget{tablet}{}\label{tablet}

A tablet has a path in zookeeper, with its action / actionlog and pid file:

{\tt 
\$ zk ls /zk/nyc/vt/tablets/0000200308
action
actionlog
pid
}

We use the action and actionlog paths for remote execution of actions. vttablet will watch that directory and launch a vtaction for every requested action.

A tablet also has a node of type Tablet:

```go
// see go/vt/topo/tablet.go for latest version
type Tablet struct \{
        Parent      TabletAlias // the globally unique alias for our replication parent - zero if this is the global master

\begin{verbatim}    // What is this tablet?
    Alias TabletAlias

    // Locaiton of the tablet
    Hostname string
    IPAddr   string

    // Named port names. Currently supported ports: vt, vts,
    // mysql.
    Portmap map[string]int

    // Tags contain freeform information about the tablet.
    Tags map[string]string

    // Information about the tablet inside a keyspace/shard
    Keyspace string
    Shard    string
    Type     TabletType

    // Is the tablet read-only?
    State TabletState

    // Normally the database name is implied by "vt_" + keyspace. I
    // really want to remove this but there are some databases that are
    // hard to rename.
    DbNameOverride string
    KeyRange       key.KeyRange
    
    // BlacklistedTables is a list of tables we're not going to serve
    // data for. This is used in vertical splits.
    BlacklistedTables []string
\end{verbatim}
\}
```

{\tt 
\$ zk cat /zk/nyc/vt/tablets/0000200308
\{
  "Alias": \{
    "Cell": "nyc",
    "Uid": 200308,
  \},
  "Parent": \{
    "Cell": "",
    "Uid": 0
  \},
  "Keyspace": "",
  "Shard": "",
  "Type": "idle",
  "State": "ReadOnly",
  "DbNameOverride": "",
  "KeyRange": \{
    "Start": "",
    "End": ""
  \}
\}
}

The Tablet object is created by 'vtctl InitTablet'. Up-to-date information (port numbers, ...) is maintained by the vttablet process. 'vtctl ChangeSlaveType' will also change the Tablet record.

\subsection{Replication Graph}\hypertarget{replication-graph}{}\label{replication-graph}

The data maintained by vt tools is as follows:
- it is stored in the global zk cell
- the master tablet alias is stored in the Shard object
- each cell then has a ShardReplication object that stores to master -\textgreater{} slave pairs.

\subsection{Serving Graph}\hypertarget{serving-graph}{}\label{serving-graph}

The serving graph for a shard is maintained in every cell that contains tablets for that shard. To get all the available keyspaces in a cell, just list the top-level cell serving graph directory:

{\tt 
\$ zk ls /zk/nyc/vt/ns
keyspace1
keyspace2
}

The python client lists that directory at startup to find all the keyspaces.

\subsubsection{SrvKeyspace}\hypertarget{srvkeyspace}{}\label{srvkeyspace}

The keyspace data is stored under /zk/

```go
// see go/vt/topo/srvshard.go for latest version
type SrvShard struct \{
        // Copied from Shard
        KeyRange    key.KeyRange
        ServedTypes []TabletType

\begin{verbatim}    // TabletTypes represents the list of types we have serving tablets
    // for, in this cell only.
    TabletTypes []TabletType

    // For atomic updates
    version int64
\end{verbatim}
\}

// A distilled serving copy of keyspace detail stored in the local
// cell for fast access. Derived from the global keyspace, shards and
// local details.
// In zk, it is in /zk/local/vt/ns/

\begin{verbatim}    // List of available tablet types for this keyspace in this cell.
    // May not have a server for every shard, but we have some.
    TabletTypes []TabletType

    // Copied from Keyspace
    ShardingColumnName string
    ShardingColumnType key.KeyspaceIdType
    ServedFrom         map[TabletType]string

    // For atomic updates
    version int64
\end{verbatim}
\}

// KeyspacePartition represents a continuous set of shards to
// serve an entire data set.
type KeyspacePartition struct \{
        // List of non-overlapping continuous shards sorted by range.
        Shards []SrvShard
\}

```

{\tt 
\$ zk cat /zk/nyc/vt/ns/rlookup
\{
  "Shards": [
    \{
      "KeyRange": \{
        "Start": "",
        "End": ""
      \},
    \}
  ],
  "TabletTypes": [
    "master",
    "rdonly",
    "replica"
  ]
\}
}

The only way to build this data is to run the following vtctl command:

{\tt 
\$ vtctl RebuildKeyspaceGraph \textless{}keyspace\textgreater{}
}

When building a new Cell, this command should be run for every keyspace.

Rebuilding a keyspace graph will:
- find all the shard names in the keyspace from looking at the children of /zk/global/vt/keyspaces/

The python client reads the nodes to find the shard map (KeyRanges, TabletTypes, ...)

\subsubsection{SrvShard}\hypertarget{srvshard}{}\label{srvshard}

The shard data is stored under /zk/

{\tt 
\$ zk cat /zk/nyc/vt/ns/rlookup/0
\{
  "KeyRange": \{
    "Start": "",
    "End": ""
  \}
\}
}

\subsubsection{EndPoints}\hypertarget{endpoints}{}\label{endpoints}

We also have per serving type data under /zk/

{\tt 
\$ zk cat /zk/nyc/vt/ns/rlookup/0/master
\{
  "entries": [
    \{
      "uid": 200274,
      "host": "nyc-db274.nyc.youtube.com",
      "port": 0,
      "named\_port\_map": \{
        "\_mysql": 3306,
        "\_vtocc": 8101,
        "\_vts": 8102
      \}
    \}
  ]
\}
}

The shard serving graph can be re-built using the 'vtctl RebuildShardGraph 

Note this will rebuild the serving graph for all cells, not just one cell.

Rebuilding a shard serving graph will:
- compute the data to write by looking at all the tablets from the replicaton graph
- write all the /zk/

The clients read the per-type data nodes to find servers to talk to. When resolving ruser.10-20.master, it will try to read /zk/local/vt/ns/ruser/10-20/master.


\chapter{Preface}
\label{chap:preface}

System Monitoring is an important subject in a IT department.
With good system monitoring tool like Xymon, it provides your IT
staff a tool to provide proactive system monitoring service.

Service outages can be detectd quickly and even prevented by monitoring
hobbit closely.

I am turning hobbit man pages into a book using latex because following reasons

\begin{itemize}
\item RTFM: Read The Fine Manpage is the most direct and efficient way of
learning a Unix subject. but it is hard if you trying to read a set manualpages.
\item  when man pages about Xymon is 58 form author Henry.
\item Problems of this approach is that manpage is not a book. it has no
Table of Content,Indexes to locate a term easily.

\end{itemize}

\section{This book is a work in progress}

I am writing this  book about hobbit from perspective of manpages.  

I am releasing this Xymon RTFM book while I am still writing it, in the hope that
it will prove useful to others.  I also hope that readers will
contribute as they see fit.

\section{Xymon Documentation Road Map}

\begin{itemize}
\item Xymon User Guide.
\item when man pages about Xymon is 58 form author Henry.
\item Problems of this approach is that manpage is not a book. it has no
Table of Content,Indexes to locate a term easily.

\end{itemize}

\section{Revision History}

\begin{itemize}
\item Henrik Storner
 \begin{enumerate}
  \item Wrote the orignial mapages in troff
 \end{enumerate}

\item T.J. Yang:
 \begin{enumerate}
  \item Import from troff sourcefile to tex
  \item fix tex file into a chapter base tex syntax.
 \end{enumerate}
\item 
\end{itemize}

\section{Colophon---this book is Free}

This book is licensed under the Open Publication License, and is
produced entirely using Free Software tools.  It is typeset with
\LaTeX{}; illustrations are drawn and rendered with
\href{http://www.inkscape.org/}{Inkscape}.

The complete source code for this book is published as a Mercurial
repository, at \url{http://hg.serpentine.com/mercurial/book}.

%%% Local Variables: 
%%% mode: latex
%%% TeX-master: "00book"
%%% End: 

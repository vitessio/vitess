% This file is a collection latex custom commands created for this book.

% for title page
\newcommand{\HRule}{\rule{\linewidth}{0.5mm}}

% Mercurial command.
\newcommand{\hbgraph}[1]{\index{\texttt{#1} hobbit graph}``\texttt{hg #1}''}

% Bug ID.
\newcommand{\bug}[1]{\index{Mercurial bug
    database!\href{http://www.selenic.com/mercurial/bts/issue#1}{bug
      ~#1}}\href{http://www.selenic.com/mercurial/bts/issue#1}{Mercurial
    bug no.~#1}}

% File name in the user's home directory.
\newcommand{\tildefile}[1]{\texttt{\~{}/#1}}

% File name.
\newcommand{\filename}[1]{\texttt{#1}}

% Directory name.
\newcommand{\dirname}[1]{\texttt{#1}}

% File name, with index entry.
% The ``s'' prefix comes from ``special''.
\newcommand{\sfilename}[1]{\index{\texttt{#1} file}\texttt{#1}}

% Directory name, with index entry.
\newcommand{\sdirname}[1]{\index{\texttt{#1} directory}\texttt{#1}}

% Mercurial extension.
\newcommand{\hgext}[1]{\index{\texttt{#1} extension}\texttt{#1}}

% Command provided by a Mercurial extension.
\newcommand{\hgxcmd}[2]{\index{\texttt{#2} command (\texttt{#1}
      extension)}\index{\texttt{#1} extension!\texttt{#2} command}``\texttt{hg #2}''}

% Mercurial command.
\newcommand{\hgcmd}[1]{\index{\texttt{#1} command}``\texttt{hg #1}''}

% Moto hobbit command
\newcommand{\motohbcmd}[1]{\index{\texttt{#1} command}``\texttt{/opt/moto/bin/#1}''}


% Mercurial command, with arguments.
\newcommand{\hgcmdargs}[2]{\index{\texttt{#1} command}``\texttt{hg #1 #2}''}

\newcommand{\tplkword}[1]{\index{\texttt{#1} template keyword}\index{template keywords!\texttt{#1}}\texttt{#1}}

\newcommand{\tplkwfilt}[2]{\index{\texttt{#1} template keyword!\texttt{#2}
    filter}\index{template filters!\texttt{#2}}\index{\texttt{#2}
    template filter}\texttt{#2}}

\newcommand{\tplfilter}[1]{\index{template
    filters!\texttt{#1}}\index{\texttt{#1} template
    filter}\texttt{#1}}

% Shell/system command.
\newcommand{\command}[1]{\index{\texttt{#1} system command}\texttt{#1}}

% Shell/system command, with arguments.
\newcommand{\cmdargs}[2]{\index{\texttt{#1} system command}``\texttt{#1 #2}''}

% Mercurial command option.
\newcommand{\hgopt}[2]{\index{\texttt{#1} command!\texttt{#2} option}\texttt{#2}}

% Mercurial command option, provided by an extension command.
\newcommand{\hgxopt}[3]{\index{\texttt{#2} command (\texttt{#1} extension)!\texttt{#3} option}\index{\texttt{#1} extension!\texttt{#2} command!\texttt{#3} option}\texttt{#3}}

% Mercurial global option.
\newcommand{\hggopt}[1]{\index{global options!\texttt{#1} option}\texttt{#1}}

% Shell/system command option.
\newcommand{\cmdopt}[2]{\index{\texttt{#1} command!\texttt{#2} option}\texttt{#2}}

% Command option.
\newcommand{\option}[1]{\texttt{#1}}

% Software package.
\newcommand{\package}[1]{\index{\texttt{#1} package}\texttt{#1}}

% Section name from a hgrc file.
\newcommand{\rcsection}[1]{\index{\texttt{hgrc} file!\texttt{#1} section}\texttt{[#1]}}

% Named item in a hgrc file section.
\newcommand{\rcitem}[2]{\index{\texttt{hgrc} file!\texttt{#1}
    section!\texttt{#2} entry}\texttt{#2}}

% hgrc file.
\newcommand{\hgrc}{\index{configuration file!\texttt{hgrc}
    (Linux/Unix)}\index{\texttt{hgrc} configuration file}\texttt{hgrc}}

% Mercurial.ini file.
\newcommand{\hgini}{\index{configuration file!\texttt{Mercurial.ini}
    (Windows)}\index{\texttt{Mercurial.ini} configuration file}\texttt{Mercurial.ini}}

% Hook name.
\newcommand{\hook}[1]{\index{\texttt{#1} hook}\index{hooks!\texttt{#1}}\texttt{#1}}

% Environment variable.
\newcommand{\envar}[1]{\index{\texttt{#1} environment
    variable}\index{environment variables!\texttt{#1}}\texttt{#1}}

% Build Environment variable.
\newcommand{\buildenvar}[1]{\index{\texttt{#1} environment
    variable}\index{variables for building!\texttt{#1}}\texttt{#1}}

% Golang Environment variable.
\newcommand{\goenvar}[1]{\index{\texttt{#1} Golang environment
    variable}\index{variables for Golang!\texttt{#1}}\texttt{#1}}

% Vitess Environment variable.
\newcommand{\vitessvar}[1]{\index{\texttt{#1} Vitess environment
    variable}\index{variables for Vitess!\texttt{#1}}\texttt{#1}}



% Vitess Concepts keywords.
\newcommand{\vconcepts}[1]{\index{\texttt{#1} Vitess Concepts
    keywords}\index{Vitess Concepts!\texttt{#1}}\texttt{#1}}

% Xymon Environment variable.
\newcommand{\xymonenvar}[1]{\index{\texttt{#1} Xymon environment
    variable}\index{Xymon environment variables!\texttt{#1}}\texttt{#1}}

% Xymon CGI script
\newcommand{\xymoncgi}[1]{\index{\texttt{#1} Xymon CGI script
        }\index{Xymon CGI scripts!\texttt{#1}}\texttt{#1}}


% Xymon CGI script
\newcommand{\xymoncfg}[1]{\index{\texttt{#1} Xymon configuration file
        }\index{Xymon configuration files!\texttt{#1}}\texttt{#1}}

% Xymon workers
\newcommand{\xymonworker}[1]{\index{\texttt{#1} Xymon worker
   }\index{Xymon Workers!\texttt{#1}}\texttt{#1}}

% Xymon daemon channels
\newcommand{\xymonchannelvar}[1]{\index{\texttt{#1} Xymon channel
    variable}\index{Xymon daemon channel variables!\texttt{#1}}\texttt{#1}}

% bb-hosts  tag.
\newcommand{\bbhosttag}[1]{\index{\texttt{#1} bb-hosts 
    tag}\index{bb-hosts  tags!\texttt{#1}}\texttt{#1}}


% BBGEN  tag.
\newcommand{\bbgentag}[1]{\index{\texttt{#1} BBGEN
    tag}\index{BBGEN  tags!\texttt{#1}}\texttt{#1}}

% Python module.
\newcommand{\pymod}[1]{\index{\texttt{#1} module}\texttt{#1}}

% Python class in a module.
\newcommand{\pymodclass}[2]{\index{\texttt{#1} module!\texttt{#2}
    class}\texttt{#1.#2}}

% Python function in a module.
\newcommand{\pymodfunc}[2]{\index{\texttt{#1} module!\texttt{#2}
    function}\texttt{#1.#2}}

% Note: blah blah.
\newsavebox{\notebox}
\newenvironment{note}%
  {\begin{lrbox}{\notebox}\begin{minipage}{0.7\textwidth}\textbf{Note:}\space}%
  {\end{minipage}\end{lrbox}\fbox{\usebox{\notebox}}}
\newenvironment{caution}%
  {\begin{lrbox}{\notebox}\begin{minipage}{0.7\textwidth}\textbf{Caution:}\space}%
  {\end{minipage}\end{lrbox}\fbox{\usebox{\notebox}}}

% Code sample, eating 4 characters of leading space.
\DefineVerbatimEnvironment{codesample4}{Verbatim}{frame=single,gobble=4,numbers=left,commandchars=\\\{\}}

% Code sample, eating 2 characters of leading space.
\DefineVerbatimEnvironment{codesample2}{Verbatim}{frame=single,gobble=2,numbers=left,commandchars=\\\{\}}

% Interaction from the examples directory.
\newcommand{\interaction}[1]{\VerbatimInput[frame=single,numbers=left,commandchars=\\\{\}]{examples/#1.lxo}}
% Example code from the examples directory.
\newcommand{\excode}[1]{\VerbatimInput[frame=single,numbers=left,commandchars=\\\{\}]{../examples/#1}}

% Graphics inclusion.
\ifpdf
  \newcommand{\grafix}[1]{\includegraphics{#1}}
\else
  \newcommand{\grafix}[1]{\includegraphics{#1.png}}
\fi

% Reference entry for a command.
\newcommand{\cmdref}[2]{\section{\hgcmd{#1}---#2}\label{cmdref:#1}\index{\texttt{#1} command}}

% Reference entry for a command option with long and short forms.
\newcommand{\optref}[3]{\subsubsection{\hgopt{#1}{--#3}, also \hgopt{#1}{-#2}}}

% Reference entry for a command option with only long form.
\newcommand{\loptref}[2]{\subsubsection{\hgopt{#1}{--#2} option}}


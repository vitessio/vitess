\section{Getting Started on Ubuntu}\label{Getting-Started}

If you run into issues or have questions, you can use our mailing list: vitess@googlegroups.com.

\subsection{Dependencies}\label{dependencies}

\begin{itemize}
\item We currently develop on Ubuntu 12.04 and 14.04.
\item You'll need some kind of Java Runtime (for ZooKeeper).
We use OpenJDK (\emph{sudo apt-get install openjdk-7-jre}).
\item \href{http://golang.org}{Go} 1.2+: Needed for building Vitess.
\item \href{https://mariadb.org/}{MariaDB}: We currently develop with version 10.0.13.
Other 10.0.x versions may also work.
\item \href{http://zookeeper.apache.org/}{ZooKeeper}: By default, Vitess
uses Zookeeper as the lock service. It is possible to plug in
something else as long as the new service supports the
necessary API functions.
\item \href{http://memcached.org}{Memcached}: Used for the rowcache.
\item \href{http://python.org}{Python}: For the client and testing.
\end{itemize}

\subsection{Building}\hypertarget{building}{}\label{building}

\begin{enumerate}
\item \href{http://golang.org/doc/install}{Install Go from golang site}.
\item \href{https://downloads.mariadb.org/}{Install MariaDB from mariadb.org site}.
You can use any installation method (src/bin/rpm/deb),
but be sure to include the client development headers (\textbf{libmariadbclient-dev}).

\item install apt package.
% \lstinputlisting[language=bash,label=src:install-apt,caption=install apt package]{src/apt-dep.bash}
\begin{codesample4}
    cd $WORKSPACE
    sudo apt-get install -y make automake  
    libtool memcached python-dev python-mysqldb  
    libssl-dev g++ mercurial git pkg-config bison curl 
\end{codesample4}

\item download vitess source.

\begin{codesample2}
    [vitess@psp vitess]$ tree -L 2 
    .
    ├── bootstrap.sh
    ├── config
    │   ├── gomysql.pc.tmpl
    │   ├── mycnf
    │   ├── zkcfg
    │   └── zk-client-dev.json
    ├── data
    │   ├── bootstrap
    │   └── test
    ├── dev.env
    ├── doc
    │   ├── Concepts.markdown
    │   ├── Contributing.markdown
    │   ├── FAQ.markdown
    │   ├── GettingStarted.markdown
    │   ├── HelicopterOverview.markdown
    │   ├── Production.markdown
    │   ├── Reparenting.markdown
    │   ├── ReplicationGraph.markdown
    │   ├── Resharding.markdown
    │   ├── SchemaManagement.markdown
    │   ├── ServingGraph.markdown
    │   ├── TestingOnARamDisk.markdown
    │   ├── Tools.markdown
    │   ├── Vision.markdown
    │   ├── Vitess2014.pdf
    │   ├── VitessComponents.png
    │   ├── VitessOverview.png
    │   ├── VitessSpectrum.png
    │   ├── Vttablet.markdown
    │   ├── VTTabletModules.png
    │   └── ZookeeperData.markdown
    ├── Dockerfile
    ├── go
    │   ├── acl
    │   ├── bson
    │   ├── bufio2
    │   ├── bytes2
    │   ├── cache
    │   ├── cgzip
    │   ├── cmd
    │   ├── db
    │   ├── event
    │   ├── executil
    │   ├── exit
    │   ├── fileutil
    │   ├── flagutil
    │   ├── hack
    │   ├── history
    │   ├── ioutil2
    │   ├── jscfg
    │   ├── memcache
    │   ├── mysql
    │   ├── netutil
    │   ├── pools
    │   ├── proc
    │   ├── README.markdown
    │   ├── rpcplus
    │   ├── rpcwrap
    │   ├── sqltypes
    │   ├── stats
    │   ├── streamlog
    │   ├── sync2
    │   ├── tb
    │   ├── terminal
    │   ├── testfiles
    │   ├── timer
    │   ├── umgmt
    │   ├── vt
    │   └── zk
    ├── java
    │   ├── bootstrap.sh
    │   ├── gorpc
    │   ├── pom.xml
    │   ├── README.markdown
    │   ├── vtgate-client
    │   ├── vtocc-client
    │   └── vtocc-jdbc-driver
    ├── LICENSE
    ├── Makefile
    ├── misc
    │   ├── git
    │   ├── gofmt-all
    │   └── parse_cover.py
    ├── py
    │   ├── cbson
    │   ├── checkers
    │   ├── io
    │   ├── net
    │   ├── setup.py
    │   ├── vtctl
    │   ├── vtdb
    │   └── zk
    ├── README.markdown
    ├── test
    │   ├── barnacle_test.py
    │   ├── binlog.py
    │   ├── checkers_test.py
    │   ├── clone.py
    │   ├── environment.py
    │   ├── environment.pyc
    │   ├── fake_zkocc_config.json
    │   ├── framework.py
    │   ├── goloadgen
    │   ├── initial_sharding_bytes.py
    │   ├── initial_sharding.py
    │   ├── java_vtgate_test_helper.py
    │   ├── keyrange_test.py
    │   ├── keyspace_test.py
    │   ├── loadgen.py
    │   ├── mysqlctl.py
    │   ├── mysql_flavor.py
    │   ├── mysql_flavor.pyc
    │   ├── primecache.py
    │   ├── queryservice_test.py
    │   ├── queryservice_tests
    │   ├── reparent.py
    │   ├── resharding_bytes.py
    │   ├── resharding.py
    │   ├── resharding_vtworker.py
    │   ├── rowcache_invalidator.py
    │   ├── schema.py
    │   ├── secure.py
    │   ├── sharded.py
    │   ├── tabletmanager.py
    │   ├── tablet.py
    │   ├── tablet.pyc
    │   ├── test_data
    │   ├── update_stream.py
    │   ├── utils.py
    │   ├── utils.pyc
    │   ├── vertical_split.py
    │   ├── vertical_split_vtgate.py
    │   ├── vertical_split_vtworker.py
    │   ├── vtctld_test.py
    │   ├── vtdb_test.py
    │   ├── vtgatev2_test.py
    │   ├── vthook-copy_snapshot_from_storage.sh
    │   ├── vthook-copy_snapshot_to_storage.sh
    │   ├── vthook-test.sh
    │   └── zkocc_test.py
    ├── third_party
    │   ├── acolyte.patch
    │   ├── go
    │   ├── mysql.patch
    │   ├── py
    │   └── zookeeper
    └── tjy
        ├── book
        ├── Makefile
        ├── mklb
        ├── osmkfiles
        └── vitess.pdf
    
    70 directories, 83 files
    [vitess@psp vitess]$ 
\end{codesample2}

\item build Vitess.

Note that the value of MYSQL\_FLAVOR is case-sensitive.
If the mysql\_config command from libmariadbclient-dev is not on the PATH,
you'll need to \emph{export VT\_MYSQL\_ROOT=/path/to/mariadb} before running bootstrap.sh,
where mysql\_config is found at /path/to/mariadb/\textbf{bin}/mysql\_config.

\begin{codesample4}
    git clone https://github.com/youtube/vitess.git src/github.com/youtube/vitess
    cd src/github.com/youtube/vitess
    export MYSQL_FLAVOR=MariaDB
    
    ./bootstrap.sh
    . ./dev.env
    make clean && make build && make site_test
\end{codesample4}
\end{enumerate}

\subsection{Testing}\hypertarget{testing}{}\label{testing}

The full set of tests included in the default \emph{make} and \emph{make test} targets
is intended for use by Vitess developers to verify code changes.

These tests simulate a small cluster by launching many servers on the local
machine, so they require a lot of resources (minimum 8GB RAM and SSD recommended).

If you are only interested in checking that Vitess is working in your
environment, you can run a set of lighter tests:

\begin{codesample2}

  make clean && make build && make site_test

\end{codesample2}

\subsubsection{Common Test Issues}\hypertarget{common-test-issues}{}\label{common-test-issues}

Many common failures come from running the full developer test suite
(\emph{make} or \emph{make test}) on an underpowered machine. If you still get
these errors with the lighter set of site tests (\emph{make site\_test}),
please let us know on the mailing list.

\paragraph{Node already exists, port in use, etc.}\hypertarget{node-already-exists-port-in-use-etc}{}\label{node-already-exists-port-in-use-etc}

Sometimes a failed test may leave behind orphaned processes.
If you use the default settings, you can find these by looking for
\emph{vtdataroot} in the command line, since every process is told to put
its files there with a command line flag. For example:
\begin{codesample4}
    pgrep -f -l '(vtdataroot|VTDATAROOT)' # list Vitess processes
    pkill -f '(vtdataroot|VTDATAROOT)'    # kill Vitess processes
\end{codesample4}


\paragraph{Too many connections to MySQL, or other timeouts}\hypertarget{too-many-connections-to-mysql-or-other-timeouts}{}\label{too-many-connections-to-mysql-or-other-timeouts}

This often means your disk is too slow. If you don't have access to an SSD,
you can try testing against a ramdisk(~\ref{Testing-On-A-Ramdisk}).

\paragraph{Connection refused to tablet, MySQL socket not found, etc.}\hypertarget{connection-refused-to-tablet-mysql-socket-not-found-etc}{}\label{connection-refused-to-tablet-mysql-socket-not-found-etc}

This could mean you ran out of RAM and a server crashed when it tried to allocate more.
Some of the heavier tests currently require up to 8GB RAM.

\paragraph{Connection refused in zkctl test}\hypertarget{connection-refused-in-zkctl-test}{}\label{connection-refused-in-zkctl-test}

This could indicate that no Java Runtime is installed.

\paragraph{Running out of disk space}\hypertarget{running-out-of-disk-space}{}\label{running-out-of-disk-space}

Some of the larger tests use up to 4GB of temporary space on disk.

\subsection{Setting up a cluster}\hypertarget{setting-up-a-cluster}{}\label{setting-up-a-cluster}

\subsection{TODO}\hypertarget{TODO}{}\label{TODO}
TODO: Expand on all sections
\begin{itemize}
\item  Setup zookeeper
\item  Start a MySql instance
\item  Start vttablet
\item  Start vtgate
\item  Write a client
\item  Test
\end{itemize}

